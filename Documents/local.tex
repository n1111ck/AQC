

% Coordenador(a) atual do Curso
\coursecoord{Alceu Heinke Frigeri}[m]
\coursecoordtitle{Coordenador de Curso}

% o nome do curso pode ser redefinido (ex. para Monografias)
%
%% \courseacronym{CCA}
%% \course{Eng. de Controle e Automação}

%\universitydivision{Escola de Engenharia}
%\university{Universidade Federal...}

%% for a report
%
%\courseundef
%\department{DELAE - Dept. de Sistemas Elétricos e Energia}
%\class{ENG100xy}{some thing else}
%\subject{Segundo trabalho da disciplina}

% o local de realização do trabalho pode ser especificado (ex. para Monografias)
% com o comando \location:
%\location{São José dos Campos}{SP}


%%%%%%%%%%%%%%%%%%%%%%%%%%%%%%%%%%%%%%
%%%%%%%%%%%%%%%%%%%%%%%%%%%%%%%%%%%%%%
%%%%%%%%%%%%%%%%%%%%%%%%%%%%%%%%%%%%%%


% Informações gerais
%
\title{Projeto de controlador para quadricópteros em entregas autônomas}

\author{Nickolas Pessim Oliveira da Fonseca}{}
\authorinfo{00302582}{}
% alguns documentos podem ter varios autores:
%\author{Flaumann}{Frida Gutenberg}
%\author{Flaumann}{Klaus Gutenberg}

% orientador
\advisor[Prof.~Dr.]{R. Sobczyk Sobrinho}{Mário}[m]
\advisorinfo{UFRGS}{Doutor pela Universidade Federal do Rio Grande do Sul -- Porto Alegre, Brasil}{email}{ramal}

% O comando \advisorwidth pode ser usado para ajustar o tamanho do campo
% destinado ao nome do orientador, de forma a evitar que ocupe mais de uma linha 
%\advisorwidth{0.55\textwidth}

% obviamente, o co-orientador é opcional
%\coadvisor[Prof.~Dr.]{do Co-orientador (se houver)}{Nome}
%\coadvisorinfo{UFRGS}{Doutor pela (Instituição onde obteve o título -- Cidade, País)}{email}{ramal}
%\coadvisorgender %% the default is [m], other option is [f]

% banca examinadora
\examiner[Prof.~Dr.]{André Perondi}{Eduardo}[m]
\examinerinfo{UFRGS}{Doutor pela Universidade Federal de Santa Catarina -- Florianópolis, Brasil}{email}{ramal}

\examiner[Prof.~Dr.]{Fetter Lages}{Walter}[m]
\examinerinfo{UFRGS}{Doutor pelo Instituto Tecnológico de Aeronáutica -- São José dos Campos, Brasil}{email}{ramal}


% suplentes da banca examinadora (apenas para alguns formulários)
\altexaminer[Prof.~Dr.]{do professor suplente I)}{(nome}
\altexaminerinfo{sigla da Instituição I onde atua}{Doutor pela (Instituição Ia onde obteve o título -- Cidade, País)}{email}{ramal}


%% resumo do trabalho (para o formulário de renovação de requerimento de matrícula.
%%
\tccbrief{algo a ser feito...}
\tccadvisorsreview{Parecer final do Orientador.}
\tcccoadvisorbrief{justificativa para ter-se um co-orientador...}

% palavras-chave
% iniciar todas com letras minúsculas, exceto no caso de abreviaturas
%
\keyword{Projeto}
\keyword{UAV}
\keyword{Entrega Autônoma}
\keyword{Automação e Controle}
\keyword{Dinâmica de voo}

% a data deve ser a da defesa; se nao especificada, são gerados
% mes e ano correntes
%\date{fevereiro}{2004}


